\documentclass{article}

% \documentclass{article}
\usepackage{xeCJK}
\usepackage{graphicx}
\usepackage{fancyhdr}
\usepackage{geometry}
\usepackage{fontspec}
\usepackage{caption}
% \usepackage[dvipsnames]{xcolor}  % 更全的色系
\usepackage{listings}
\usepackage{../code/rust/listings-rust}
\usepackage[most]{tcolorbox}
\usepackage{colortbl} % 提供表格着色功能
\usepackage{array}    % 提供表格调整功能


\newcommand{\myname}{陈楷文}
\newcommand{\mystudentid}{202232110214}
\newcommand{\myclass}{软件工程(中外合作办学)222}
% \newcommand{\myfilename}{实验一}






% 设置页面边距
\geometry{
    top=2cm,    % 上边距
    bottom=2cm, % 下边距
    left=2cm,   % 左边距
    right=2cm   % 右边距
}
% 设置英语字体
% 设置专门的字体,使用TTF文件中的字体
\setmainfont{Georgia}[
    Path = ../ttf/,
    Extension = .ttf,
]

\setsansfont{FiraCode Nerd Font}
\setmonofont{UbuntuMono Nerd Font Mono}

% 设置中文字体
\setCJKmainfont{WenQuanYi Zen Hei Mono}
\setCJKsansfont{xiawuzhisong}[
    Path = ../ttf/,
    Extension = .ttf,
]
\setCJKmonofont{sarasa-term-sc-nerd-regular}[
    Path = ../ttf/,
    Extension = .ttf,
]

% 设置专门的字体,使用TTF文件中的字体
\newfontfamily\myfont{sarasa-term-sc-nerd-regular}[
    Path = ../ttf/,
    Extension = .ttf,
]



\newcommand{\myfilename}{黑盒测试-边界值测试法}
\newcommand{\rust}{/home/kevin/Documents/rust/tesxsthw1/src/main.rs}
\newcommand{\bash}{/home/kevin/Documents/rust/tesxsthw1/src/test.sh}
\newcommand{\lightgreentitle}{rust code}
\newcommand{\moonstonebluetitle}{bash code}
\newcommand{\moonstoneblue}{\lstinputlisting[
  language=Bash% 
, breaklines = true% 
, style = boxed%
, escapeinside={<TeX>}{</TeX>}%
]{\bash}
}
\newcommand{\lightgreen}{\lstinputlisting[language=Rust, style=boxed, breaklines = true, escapeinside={<TeX>}{</TeX>}]{\rust}
}

\newcommand{\blue}{\begin{tabular}{|l|c|c|c|r|}
%\hline
%\multicolumn{3}{|c|}{标题} \\ % 合并单元格
\hline
  a & b & c & 测试输出 & 预期输出\\
\hline

  \input{/home/kevin/Documents/rust/tesxsthw1/src/test.log}

\end{tabular}}
\newcommand{\bluetitle}{三角形测试数据}

\begin{document}



\pagestyle{fancy}
\setlength{\headsep}{0.5cm} % 设置距离顶部的长度
\fancyhf{}
% \renewcommand{\headrulewidth}{0pt} % 去掉页眉下方的横线
\fancyhead[C]{%
    \parbox{\textwidth}{%
        \includegraphics[width=12pt]{../img/zjnu.jpg}%\hfill
        {\myfont 良好的测试是软件质量的有效保证} % 使用自定义字体
    }%
} % 在页眉中插入图片和文字

% 自定义页脚
%\fancyfoot[C]{Page \thepage} % 页码居中显示在页脚中
\fancyfoot[L]{- \thepage\ -} % 在左边显示页码
\fancyfoot[R]{浙江师范大学计算机科学与技术学院} % 在右边显示自定义文本

 % 页眉页脚


\begin{center}
  \vspace*{\fill} % 将内容向下推至页面底部

    \begin{huge}
      \textbf{《软件质量保证与测试》}\\
      \textbf{实验报告}\\
    \end{huge}

    \vspace{3cm}

    \includegraphics[width=150pt]{../img/zjnu.jpg} % 替换 "your_image_file" 为你的图片文件名
    %\caption{图片说明}

    \vspace{3cm}

    \newlength{\linelength}
    \setlength{\linelength}{5cm} % 设置固定长度为3厘米
    \newlength{\itemlength}
    \setlength{\itemlength}{2cm} % 设置固定长度为3厘米

    \makebox[\textwidth]{\hbox to \itemlength{姓名:\hfill}  \underline{\hbox to \linelength{\hskip 0.5cm \myname\hfill}}}
    \makebox[\textwidth]{\hbox to \itemlength{学号:\hfill}  \underline{\hbox to \linelength{\hskip 0.5cm \mystudentid\hfill}}}
    \makebox[\textwidth]{\hbox to \itemlength{班级:\hfill}  \underline{\hbox to \linelength{\hskip 0.5cm \myclass\hfill}}}
    \makebox[\textwidth]{\hbox to \itemlength{实验名称:\hfill} \underline{\hbox to \linelength{\hskip 0.5cm \myfilename\hfill}}}

  \vspace*{\fill} % 将内容向上推至页面顶部
\end{center}

\newpage
 % 首页

\title{实验一:黑盒测试-边界值测试法}
\author{陈楷文}
\date{\today}

\maketitle

\section{[实验环境]}
\begin{itemize}
    \item 操作系统:Linux
    \item 程序设计语言:Rust
    \item 脚本设计语言:Bash
\end{itemize}
\section{[实验类型]}

黑盒测试 \\
边界值测试 \\

\section{[实验目的]}
\begin{itemize}
    \item 认识黑盒测试方法中边界值分析测试法原理
    \item 掌握黑盒测试方法中边界值分析测试法过程
\end{itemize}

\section{[实验内容]}
\begin{enumerate}
    \item 编写三角形程序
    \item 编写三角形程序测试脚本
    \item 编写NextDay程序
    \item 编写NextDay程序测试脚本
    \item 运行测试
    \item 分析测试结果
\end{enumerate}
\section{[问题描述]}
\subsection{三角形问题}

问题描述:三角形问题接受三个整数,a、b和c作为输入,用作三角形的边。程序的输出是由这三条边确定的三角形类型:等边三角形、等腰三角形、不等边三角形或非三角形。\\

作为输入:三角形的三条边必须满足如下条件:\\

C1:1<=a<=100\\

C2:1<=b<=100\\

C3:1<=c<=100\\

C4:a<b+c\\

C5:b<a+c\\

C6:c<a+b\\

\subsection{NextDay问题}

问题描述:NextDate是一个由三个变量(月份、日期和年份)的函数。函数返回输入日期后边的那个日期。\\

作为输入:变量月份、日期和年都具有整数值,满足以下条件。\\

C1:1<=月份<=12\\

C2:1<=日期<=31\\

C3:1912<=年<=2050\\

\section{[算法描述]}
\subsection{三角形程序}
use std::env;:导入了env模块,用于处理命令行参数。\\
fn main() { ... }:程序的入口函数。\\
let args: Vec<String> = env::args().collect();:将命令行参数收集到一个字符串向量中。\\
let (a, b, c) = match parse\_arguments(\&args) { ... }:调用parse\_arguments函数解析命令行参数,并将解析结果绑定到变量(a, b, c)中。\\
fn parse\_arguments(args: \&[String]) -> Result<(u32, u32, u32), String> { ... }:解析命令行参数的函数。它接受一个字符串切片作为参数,返回一个Result枚举,其中Ok包含三个边长,Err包含错误信息。\\
for i in 1..args.len() { ... }:遍历命令行参数。\\
match args[i].as\_str() { ... }:匹配当前命令行参数的字符串值。\\
-a, -b, -c:检查是否遇到了命令行参数-a、-b或-c。\\
Some(value):如果解析成功,返回一个包含解析后的值的Some枚举。\\
Some(args[i + 1].parse().map\_err(|\_| "边长a必须是一个有效的整数")?)?:将下一个参数解析为整数,如果解析失败,则返回一个包含错误信息的Err枚举。\\
is\_triangle函数:检查三条边是否能构成三角形。\\
输出结果:根据判断的结果输出对应的信息,例如等边三角形、等腰三角形、不等边三角形或非三角形。\\

\subsection{NextDay程序}

首先,程序使用了std::env模块来获取命令行参数。通过env::args()函数获取参数列表,并将其收集到一个Vec<String>类型的变量args中。\\
然后,程序定义了main()函数作为程序的入口点。在main()函数中,它遍历命令行参数列表,解析出年份、月份和日期,并存储到相应的变量中。\\
接下来,程序调用了is\_valid\_date()函数来检查输入的日期是否有效。这个函数会检查年份是否在1912到2050之间,月份是否在1到12之间,日期是否在1到31之间。如果日期无效,程序会打印错误消息并退出。\\
如果日期有效,程序就会调用next\_date()函数来计算下一个日期。这个函数会根据当前日期的年、月、日来计算下一个日期,并考虑闰年和月底的情况。\\
最后,程序打印出计算得到的下一个日期。\\

\section{[测试案例]}
\tcbset{colframe = blue!50!black, colback = white,
        colupper = red!50!black, fonttitle = \bfseries,
        nobeforeafter, center title}

\tcbox[left = 0mm, right = 0mm, top = 0mm, bottom = 0mm, boxsep = 0mm,
      toptitle = 0.5mm, bottomtitle = 0.5mm, title = {\bluetitle}]
  {\arrayrulecolor{blue!50!black}\blue}
%



\renewcommand{\blue}{\begin{tabular}{|l|c|c|c|r|}
  %\hline
  %\multicolumn{3}{|c|}{标题} \\ % 合并单元格
  \hline
  y & m & d & 测试输出 & 预期输出\\
  \hline

  \input{/home/kevin/Documents/rust/tesxsthw2/src/test.log}

\end{tabular}}
\renewcommand{\bluetitle}{NextDay测试数据}

\tcbset{colframe = blue!50!black, colback = white,
        colupper = red!50!black, fonttitle = \bfseries,
        nobeforeafter, center title}

\tcbox[left = 0mm, right = 0mm, top = 0mm, bottom = 0mm, boxsep = 0mm,
      toptitle = 0.5mm, bottomtitle = 0.5mm, title = {\bluetitle}]
  {\arrayrulecolor{blue!50!black}\blue}
%



\section{[测试结果分析]}

测试结果符合预期,基本可以认定程序正确\\

\section{[实验总结]}

在进行黑盒测试的边界值测试法实验后,我得出了一些结论。边界值测试法是一种测试方法,着重于测试输入值的边界情况,因为这些情况通常更容易导致程序错误。在实验中,我使用了这种方法来测试程序的各种输入情况,并且得出了以下几点总结。\\

首先,边界值测试法能够有效地发现程序中的潜在错误。通过测试输入值的边界情况,我发现了一些在正常情况下可能被忽视的问题。例如,当输入值处于边界上时,程序可能会产生意料之外的行为,比如溢出或者未处理的异常情况。因此,边界值测试法可以帮助我们发现这些潜在的问题,从而提高程序的质量和稳定性。\\

其次,边界值测试法有助于提高测试效率。相比于随机选择输入值进行测试,使用边界值测试法可以更有针对性地选择测试用例,从而更有效地覆盖程序的各种可能情况。这样一来,我们可以在更短的时间内发现更多的问题,提高测试的效率和准确性。\\

另外,边界值测试法也有一些局限性。虽然这种方法可以有效地发现一些特定类型的错误,但并不能覆盖所有可能的情况。有些程序错误可能不仅仅与输入值的边界有关,还与输入值的组合、程序的逻辑等因素有关。因此,在进行测试时,我们仍需要结合其他测试方法,如等价类划分、状态转换测试等,来全面地覆盖程序的各种情况。\\

综上所述,边界值测试法是一种有效的黑盒测试方法,可以帮助我们发现程序中的潜在错误,并提高测试效率。然而,在使用这种方法时,我们也需要注意其局限性,结合其他测试方法来进行综合测试,以确保程序的质量和稳定性。\\

\section{[附:程式源码]}

\definecolor{lightgreen}{rgb}{0.56, 0.93, 0.56}

\begin{tcolorbox}[
    enhanced,
    attach boxed title to top left={xshift=6mm,yshift=-3mm},
    colback=lightgreen!20,
    colframe=lightgreen,
    colbacktitle=lightgreen,
    title=\lightgreentitle,
    fonttitle=\bfseries\color{black},
    boxed title style={size=small,colframe=lightgreen,sharp corners},
    sharp corners,
    breakable,
]

\lightgreen

\end{tcolorbox}



\definecolor{moonstoneblue}{rgb}{0.45, 0.66, 0.76}

\begin{tcolorbox}[
    enhanced,
    attach boxed title to top left={xshift=6mm,yshift=-3mm},
    colback=moonstoneblue!20,
    colframe=moonstoneblue,
    colbacktitle=moonstoneblue,
    title=\moonstonebluetitle,
%    title=Welcome to \TeX{} -- \LaTeX{} Stack Exchange!,
    fonttitle=\bfseries\color{black},
    boxed title style={size=small,colframe=moonstoneblue,sharp corners},
    sharp corners,
    breakable,
]

\moonstoneblue

\end{tcolorbox}



\renewcommand{\rust}{/home/kevin/Documents/rust/tesxsthw2/src/main.rs}
\renewcommand{\bash}{/home/kevin/Documents/rust/tesxsthw2/src/test.sh}

\definecolor{lightgreen}{rgb}{0.56, 0.93, 0.56}

\begin{tcolorbox}[
    enhanced,
    attach boxed title to top left={xshift=6mm,yshift=-3mm},
    colback=lightgreen!20,
    colframe=lightgreen,
    colbacktitle=lightgreen,
    title=\lightgreentitle,
    fonttitle=\bfseries\color{black},
    boxed title style={size=small,colframe=lightgreen,sharp corners},
    sharp corners,
    breakable,
]

\lightgreen

\end{tcolorbox}



\definecolor{moonstoneblue}{rgb}{0.45, 0.66, 0.76}

\begin{tcolorbox}[
    enhanced,
    attach boxed title to top left={xshift=6mm,yshift=-3mm},
    colback=moonstoneblue!20,
    colframe=moonstoneblue,
    colbacktitle=moonstoneblue,
    title=\moonstonebluetitle,
%    title=Welcome to \TeX{} -- \LaTeX{} Stack Exchange!,
    fonttitle=\bfseries\color{black},
    boxed title style={size=small,colframe=moonstoneblue,sharp corners},
    sharp corners,
    breakable,
]

\moonstoneblue

\end{tcolorbox}





\end{document}

