\documentclass{article}

% \documentclass{article}
\usepackage{xeCJK}
\usepackage{graphicx}
\usepackage{fancyhdr}
\usepackage{geometry}
\usepackage{fontspec}
\usepackage{caption}
% \usepackage[dvipsnames]{xcolor}  % 更全的色系
\usepackage{listings}
\usepackage{../code/rust/listings-rust}
\usepackage[most]{tcolorbox}
\usepackage{colortbl} % 提供表格着色功能
\usepackage{array}    % 提供表格调整功能


\newcommand{\myname}{陈楷文}
\newcommand{\mystudentid}{202232110214}
\newcommand{\myclass}{软件工程(中外合作办学)222}
% \newcommand{\myfilename}{实验一}






% 设置页面边距
\geometry{
    top=2cm,    % 上边距
    bottom=2cm, % 下边距
    left=2cm,   % 左边距
    right=2cm   % 右边距
}
% 设置英语字体
% 设置专门的字体,使用TTF文件中的字体
\setmainfont{Georgia}[
    Path = ../ttf/,
    Extension = .ttf,
]

\setsansfont{FiraCode Nerd Font}
\setmonofont{UbuntuMono Nerd Font Mono}

% 设置中文字体
\setCJKmainfont{WenQuanYi Zen Hei Mono}
\setCJKsansfont{xiawuzhisong}[
    Path = ../ttf/,
    Extension = .ttf,
]
\setCJKmonofont{sarasa-term-sc-nerd-regular}[
    Path = ../ttf/,
    Extension = .ttf,
]

% 设置专门的字体,使用TTF文件中的字体
\newfontfamily\myfont{sarasa-term-sc-nerd-regular}[
    Path = ../ttf/,
    Extension = .ttf,
]



\newcommand{\myfilename}{实验一:黑盒测试}
\newcommand{\rust}{/home/kevin/Documents/rust/tesxsthw1/src/main.rs}
\newcommand{\bash}{/home/kevin/Documents/rust/tesxsthw1/src/test.sh}
\newcommand{\lightgreentitle}{rust code}
\newcommand{\moonstonebluetitle}{bash code}
\newcommand{\moonstoneblue}{\lstinputlisting[
  language=Bash% 
, breaklines = true% 
, style = boxed%
, escapeinside={<TeX>}{</TeX>}%
]{\bash}
}
\newcommand{\lightgreen}{\lstinputlisting[language=Rust, style=boxed, breaklines = true, escapeinside={<TeX>}{</TeX>}]{\rust}
}

\newcommand{\blue}{\begin{tabular}{|l|c|c|c|r|}
%\hline
%\multicolumn{3}{|c|}{标题} \\ % 合并单元格
\hline
  a & b & c & 测试输出 & 预期输出\\
\hline

  \input{/home/kevin/Documents/rust/tesxsthw1/src/test.log}

\end{tabular}}
\newcommand{\bluetitle}{table1}

\begin{document}


\pagestyle{fancy}
\setlength{\headsep}{0.5cm} % 设置距离顶部的长度
\fancyhf{}
% \renewcommand{\headrulewidth}{0pt} % 去掉页眉下方的横线
\fancyhead[C]{%
    \parbox{\textwidth}{%
        \includegraphics[width=12pt]{../img/zjnu.jpg}%\hfill
        {\myfont 良好的测试是软件质量的有效保证} % 使用自定义字体
    }%
} % 在页眉中插入图片和文字

% 自定义页脚
%\fancyfoot[C]{Page \thepage} % 页码居中显示在页脚中
\fancyfoot[L]{- \thepage\ -} % 在左边显示页码
\fancyfoot[R]{浙江师范大学计算机科学与技术学院} % 在右边显示自定义文本

 % 页眉页脚


\begin{center}
  \vspace*{\fill} % 将内容向下推至页面底部

    \begin{huge}
      \textbf{《软件质量保证与测试》}\\
      \textbf{实验报告}\\
    \end{huge}

    \vspace{3cm}

    \includegraphics[width=150pt]{../img/zjnu.jpg} % 替换 "your_image_file" 为你的图片文件名
    %\caption{图片说明}

    \vspace{3cm}

    \newlength{\linelength}
    \setlength{\linelength}{5cm} % 设置固定长度为3厘米
    \newlength{\itemlength}
    \setlength{\itemlength}{2cm} % 设置固定长度为3厘米

    \makebox[\textwidth]{\hbox to \itemlength{姓名:\hfill}  \underline{\hbox to \linelength{\hskip 0.5cm \myname\hfill}}}
    \makebox[\textwidth]{\hbox to \itemlength{学号:\hfill}  \underline{\hbox to \linelength{\hskip 0.5cm \mystudentid\hfill}}}
    \makebox[\textwidth]{\hbox to \itemlength{班级:\hfill}  \underline{\hbox to \linelength{\hskip 0.5cm \myclass\hfill}}}
    \makebox[\textwidth]{\hbox to \itemlength{实验名称:\hfill} \underline{\hbox to \linelength{\hskip 0.5cm \myfilename\hfill}}}

  \vspace*{\fill} % 将内容向上推至页面顶部
\end{center}

\newpage
 % 首页

\section{这是一个段落标题}

\begin{table}[htbp]
  \centering
  \caption{示例表格}
  \label{tab:example}
  \begin{tabular}{|c|c|c|}
    \hline
    姓名 & 分数 & 好\\
    \hline
    小明 & 85 & yes \\
    小红 & 92 & no \\
    小华 & 78 & yes \\
    \hline
  \end{tabular}
\end{table}

% \lstinputlisting[language=Rust, style=boxed, breaklines = true, escapeinside={<TeX>}{</TeX>}]{\rust}

\definecolor{lightgreen}{rgb}{0.56, 0.93, 0.56}

\begin{tcolorbox}[
    enhanced,
    attach boxed title to top left={xshift=6mm,yshift=-3mm},
    colback=lightgreen!20,
    colframe=lightgreen,
    colbacktitle=lightgreen,
    title=\lightgreentitle,
    fonttitle=\bfseries\color{black},
    boxed title style={size=small,colframe=lightgreen,sharp corners},
    sharp corners,
    breakable,
]

\lightgreen

\end{tcolorbox}





% \end{lstlisting}
% \end{tcolorbox}

\definecolor{moonstoneblue}{rgb}{0.45, 0.66, 0.76}

\begin{tcolorbox}[
    enhanced,
    attach boxed title to top left={xshift=6mm,yshift=-3mm},
    colback=moonstoneblue!20,
    colframe=moonstoneblue,
    colbacktitle=moonstoneblue,
    title=\moonstonebluetitle,
%    title=Welcome to \TeX{} -- \LaTeX{} Stack Exchange!,
    fonttitle=\bfseries\color{black},
    boxed title style={size=small,colframe=moonstoneblue,sharp corners},
    sharp corners,
    breakable,
]

\moonstoneblue

\end{tcolorbox}


\tcbset{colframe = blue!50!black, colback = white,
        colupper = red!50!black, fonttitle = \bfseries,
        nobeforeafter, center title}

\tcbox[left = 0mm, right = 0mm, top = 0mm, bottom = 0mm, boxsep = 0mm,
      toptitle = 0.5mm, bottomtitle = 0.5mm, title = {\bluetitle}]
  {\arrayrulecolor{blue!50!black}\blue}
%



% \end{tcolorbox}


\end{document}

